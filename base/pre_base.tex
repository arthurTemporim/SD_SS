%Centralizar verticalmente.
\newenvironment{midpage}{\vspace*{\fill}}{\vspace*{\fill}}
%Centralizar horizontalmente.
\newenvironment{midline}{\hspace*{\fill}}{\hspace*{\fill}}
\documentclass[12pts]{article}
\usepackage[utf8]{inputenc} 
\title{
	Prática da Eletônica Digital 1 (167983)
	\singlespace
	Turma FF
	\singlespacing
	\begin{midpage}
	\begin {large}
		Pré-relatório Experimento 3
		\singlespace
		Circuitos Somadores e Subtratores
	\end {large}
	\end{midpage}
}
\date{Abril 20, 2016}
\usepackage{indentfirst}
\usepackage{setspace}
\usepackage{verbatim}
\usepackage[pdftex]{hyperref}
\usepackage{graphicx}

\begin{document}
\maketitle	
%\vspace{100 mm}
\begin{center}

\begin{tabular}{|c|l|r|}
\hline
Nome & Matrícula & Assinatura\\
\hline
Arthur Temporim & 140016759 & \\
\hline	
Eduardo Nunes& 140056149 & \\
\hline	
\end{tabular}

\end{center}



\pagebreak
\section{Pesquisa Bibliográfica}
%Colocar aqui a Pesquisa Bibliográfica

\section{Projeto 1 Complemento de 1}

\subsection{Diagrama esquemático}
%Tire o \iffalse e \fi e coloque o nome da imagem.

\iffalse
\begin{figure}[!htb]
  \centering
  \includegraphics[scale=0.6	]{simulacao1}
  \caption{Simulação 1. Circuito simulado em protheus 8.1 pro.}
  \label{figRotulo}
\end{figure}
\fi

\subsection{Tabela Verdade}

%Só colocar os valores reais da tabela verdade
\begin{center}
	\begin{tabular}{|c|l|r|r|r|r|}
	\hline
	A & B & C & S\\
	\hline
	0 & 0 & 0 & 0\\
	\hline
	0 & 0 & 1 & 0\\
	\hline
	0 & 1 & 0 & 0\\
	\hline
	0 & 1 & 1 & 1\\
	\hline
	1 & 0 & 0 & 0\\
	\hline
	1 & 0 & 1 & x\\
	\hline
	1 & 1 & 0 & 1\\
	\hline
	1 & 1 & 1 & 0\\
	\hline
	\end{tabular}
\end{center}

\section{Projeto 2 Overflow}

\subsection{Diagrama esquemático}

\subsection{Tabela Verdade}


\section{Projeto 3 Somador/Subtrator}

\subsection{Diagrama esquemático}

\subsection{Tabela Verdade}


\section{Referências Bibliográficas}

\end{document}

\ifflase
Coloque para pular uma página:
\newpage
\fi