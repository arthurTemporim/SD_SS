%Centralizar verticalmente.
\newenvironment{midpage}{\vspace*{\fill}}{\vspace*{\fill}}
%Centralizar horizontalmente.
\newenvironment{midline}{\hspace*{\fill}}{\hspace*{\fill}}
\documentclass[12pts]{article}
\usepackage[utf8]{inputenc} 
\title{
	Sistemas Digitais (167983)
	\singlespace
	Turma FF
	\singlespacing
	\begin{midpage}
	\begin {large}
		Relatório Experimento X
		\singlespace
		Título
	\end {large}
	\end{midpage}
}
\date{Abril 11, 2016}
\usepackage{indentfirst}
\usepackage{setspace}
\usepackage{verbatim}
\usepackage[pdftex]{hyperref}
\usepackage{graphicx}

\begin{document}
\maketitle	
%\vspace{100 mm}
\begin{center}

\begin{tabular}{|c|l|r|}
\hline
Nome & Matrícula & Assinatura\\
\hline
Arthur Temporim & 140016759 & \\
\hline	
Eduardo Nunes & 140056189 & \\
\hline	
\end{tabular}

\end{center}


\newpage

\section{Sumário}

\begin{itemize}
\item Introdução
\item Objetivos
\item Materiais Utilizados
\item Procedimentos
\subitem Montagem do 1º Circuito
\subsubitem Tabela verdade
\subsubitem Ordem de montagem do circuito
\subitem Montagem do 2º Circuito
\subsubitem Tabela verdade
\subsubitem Ordem de montagem do circuito
\item Discussão
\subitem Caracterização de CIs
\item Conclusões 
\item Referências Bibliograficas
\item Tabelas de conexão
\item Diagramas esquemáticos
\end{itemize}

\newpage


\section{Introdução}

\singlespacing
\section{Objetivos}

\section{Materiais Utilizados}


\begin{tabular}{|c|r|}
\hline
Material Utilizado & Quantidade\\
\hline
Cabo Banana-Banana & 2  \\
\hline
Fios de cobre & x \\
\hline
Cabo coaxial & 3  \\
\hline
CI 74HC00   & 1 \\
\hline
CI 74LS00   & 1 \\
\hline
Protoboard & 2 \\
\hline
Fonte de tensão MPL-3305M & 1 \\
\hline	
Multímetro Digital  & 1 \\
\hline
Gerador de funçoes iCEL modelo GV-2002 & 1 \\
\hline
Osciloscopio BK 2530 & 1 \\
\hline
\end{tabular}
\singlespacing

\section{Procedimentos}

\subsection{Montagem do 1º Circuito}

\subsubsection{Tabela verdade}
\iffalse
\begin{center}
	\begin{tabular}{|c|l|r|r|}
		\hline
		A&S\\
		\hline
		L&H\\
		\hline
		H&L\\
		\hline
	\end{tabular}
\end{center}
\fi
\subsubsection{Ordem de montagem do circuito}
\iffalse
\begin{itemize}
\item Primeiramente a fonte de alimentação foi escolhida e foi fixada o mesmo GND para as duas 
saidas.
\item A protoboard foi alimentada.
\item CI NAND (74LS00) foi posicionado e alimentado.
\item Os pinos 1 e 2 do CI foram ligados e a fonte de alimentação variável foi conectada no
primeiro pino.
\item O pino 3 (saida da porta NAND) foi conectado ao multímetro digital de forma a verificar
os valores de saída da porta lógica.
\item Foi aumentado o valor da fonte variável de na escala de  0,5V.
\item Os valores de saída foram anotados.
\item O CI NAND (74LS00), da familia TTL foi substituido pelo CI NAND (74HC00),
da familia CMOS, para fazer novas anotaçoes.
\item O circuito eletrônico teve comportamento de acordo com o desejado.
\end{itemize}
\fi

\subsection{Montagem do 2º Circuito (setup para tempos)}

\subsubsection{Tabela verdade}

\subsubsection{Ordem de montagem do circuito}

\section{Discussão}

\subsection{Circuito 1}

\subsection{Circuito 2}

\section{Conclusões}

\section{Referências Bibliográficas}

Laboratório de Sistemas Digitais I 2015.1 professores Fabiano Soares, 
José Felício Marcus Chaffim Renato Lopes UnB - FGA Março de 2015.

\section{Tabelas de Conexão}

%Exemplo de tabela:
\iffalse
\singlespacing
\begin{tabular}{|c|l|r|r|}
\hline
SETUP PARA CURVA & CI 7400\\
\hline
De & Para\\
\hline
PINO 1 & Fonte variável e Pino 2\\
\hline
PINO 2 & Pino 1\\
\hline
PINO 3 & Multímetro\\
\hline
PINO 4 & NULL\\
\hline
PINO 5 & NULL\\
\hline
PINO 6 & NULL\\
\hline
PINO 7 & GND\\
\hline
PINO 8 & NULL\\
\hline
PINO 9 & NULL\\
\hline
PINO 10 & NULL\\
\hline
PINO 11 & NULL\\
\hline
PINO 12 & NULL\\
\hline
PINO 13 & NULL\\
\hline
PINO 14 & VCC\\
\hline
\end{tabular}
\singlespacing
\fi

\newpage
\section{Diagramas Esquemáticos}

\subsection{1º Circuito}

%Exemplo de imagem
\iffalse
\begin{figure}[!htb]
  \centering
  \includegraphics[scale=0.3	]{nome_da_imagem}
  \caption{Descrição}
  \label{figRotulo}
\end{figure}
\fi

\newpage
\subsection{2º Circuito}

\end{document}
