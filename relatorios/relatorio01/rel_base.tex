%Centralizar verticalmente.
\newenvironment{midpage}{\vspace*{\fill}}{\vspace*{\fill}}
%Centralizar horizontalmente.
\newenvironment{midline}{\hspace*{\fill}}{\hspace*{\fill}}
\documentclass[12pts]{article}
\usepackage[utf8]{inputenc} 
\title{
	Sistemas Digitais (119466)
	Unb - Gama
	\singlespace
	Turma E
	\singlespacing
	\begin{midpage}
	\begin {large}
		Relatório Experimento 1
		\singlespace
		Portas Lógicas
	\end {large}
	\end{midpage}
}
\date{Agosto 20, 2016}
\usepackage{indentfirst}
\usepackage{setspace}
\usepackage{verbatim}
\usepackage[pdftex]{hyperref}
\usepackage{graphicx}
\begin{document}
\maketitle	
%\vspace{100 mm}
\begin{center}

\begin{tabular}{|c|l|r|}
\hline
Nome & Matrícula & Assinatura\\
\hline
Arthur Temporim & 140016759 & \\
\hline	
Eduardo Nunes & 140056189 & \\
\hline	
\end{tabular}

\end{center}


\newpage

\section{Sumário}

\begin{itemize}
	\item Introdução
	\singlespacing
	\item Parte Experimental
	\singlespacing
	\item Discussão
	\singlespacing
	\item Conclusões 
	\singlespacing
	\item Referências Bibliograficas
	\singlespacing
	\item Diagramas esquemáticos
\end{itemize}

\newpage


\section{Introdução}
\iffalse
Introdução, indicando a delimitação do tema, apresentando a justificativa descrevendo o propósito do relatório.
\fi

\section{Procedimentos}
\iffalse
Parte Experimental, descrevendo os passos realizados, dificuldades e soluções para os problemas encontrados. Aqui, deve-se apresentar uma descrição dos resultados encontrados em forma de figuras, gráficos e tabelas.
\fi

\section{Discussão}
\iffalse
Discussão sobre os resultados encontrados, comentando detalhadamente as medições realizadas e dando a devida interpretação destas, informando se os objetivos da experimento foram alcançados. Esta é uma das partes mais importantes do relatório: aqui, há oportunidade para expressar os conhecimentos adquiridos na prática e fazer a interrelação com os fundamentos teóricos.
\fi

\section{Conclusões}
\iffalse
Conclusões, mostrando os êxitos e eventuais problemas encontrados na realização do experimento, indicando as limitações, apresentando recomendações e/ou sugestões.
\fi

\section{Referências Bibliográficas}
\iffalse
Referencias Bibliográficas, relacionadas e citadas de acordo com as normas da ABNT.
\fi
Laboratório de Sistemas Digitais I 2015.1 professores Fabiano Soares, 
José Felício Marcus Chaffim Renato Lopes UnB - FGA Março de 2015.

\section{Diagramas Esquemáticos}
\iffalse
Diagramas Esquemáticos. Todos os diagramas devem ser inseridos ao final do relatório em páginas separadas do texto, indicando a identificação do circuito, autor, revisor, versão e datas relevantes.
\fi

%Exemplo de tabela:
\iffalse
\singlespacing
\begin{tabular}{|c|l|r|r|}
\hline
SETUP PARA CURVA & CI 7400\\
\hline
De & Para\\
\hline
PINO 1 & Fonte variável e Pino 2\\
\hline
PINO 2 & Pino 1\\
\hline
PINO 3 & Multímetro\\
\hline
PINO 4 & NULL\\
\hline
PINO 5 & NULL\\
\hline
PINO 6 & NULL\\
\hline
PINO 7 & GND\\
\hline
PINO 8 & NULL\\
\hline
PINO 9 & NULL\\
\hline
PINO 10 & NULL\\
\hline
PINO 11 & NULL\\
\hline
PINO 12 & NULL\\
\hline
PINO 13 & NULL\\
\hline
PINO 14 & VCC\\
\hline
\end{tabular}
\singlespacing
\fi

\newpage
\end{document}

%Exemplo de imagem
\iffalse
\begin{figure}[!htb]
  \centering
  \includegraphics[scale=0.3	]{nome_da_imagem}
  \caption{Descrição}
  \label{figRotulo}
\end{figure}
\fi

% Exemplo de tabela.
\iffalse
\begin{tabular}{|c|r|}
\hline
Material Utilizado & Quantidade\\
\hline
Cabo Banana-Banana & 2  \\
\hline
Fios de cobre & x \\
\hline
Cabo coaxial & 3  \\
\hline
CI 74HC00   & 1 \\
\hline
CI 74LS00   & 1 \\
\hline
Protoboard & 2 \\
\hline
Fonte de tensão MPL-3305M & 1 \\
\hline	
Multímetro Digital  & 1 \\
\hline
Gerador de funçoes iCEL modelo GV-2002 & 1 \\
\hline
Osciloscopio BK 2530 & 1 \\
\hline
\end{tabular}
\singlespacing
\fi